
\frame{
	\frametitle{Install/Run Shiny apps}

	\textbf{Shiny is an R package that makes it easy to build apps}

	Install Shiny  
	\begin{itemize}
		\item \texttt{install.packages("shiny")}
	\end{itemize}

	Running a Shiny app in a directory called \texttt{my-app}
		\begin{enumerate}
			\item library(shiny)
			\item runApp(`my-app')
		\end{enumerate}

	R session will be busy while Shiny app is active
	\begin{itemize}
		\item Run another Shiny app $\rightarrow$ Open another R session 
		\item Get R session back $\rightarrow$ Hit \texttt{esc}
	\end{itemize}


	% http://shiny.rstudio.com/
	% http://shiny.rstudio.com/tutorial/lesson1/
	
}


\frame{
	\frametitle{Examples of Shiny apps}

	Examples built into Shiny package: Run from R or view online  
	\begin{itemize}
		\item http://shiny.rstudio.com/tutorial/lesson1/\#Go Further
		\item Each demonstrates a feature of Shiny apps
		\item Examples open in `showcase' mode with the ui.R and server.R
		scripts in the display.
	\end{itemize}

	 \vspace*{\baselineskip} % --------------------------------------------------


	Gallery of examples available online: 
	\begin{itemize}
		\item http://shiny.rstudio.com/gallery/
		\item Contains useful examples to learn from
	\end{itemize}

	% http://shiny.rstudio.com/tutorial/lesson1/


}


\frame{
	\frametitle{References/Resources}

	shiny.rstudio.com/* 
	\begin{itemize}
		\item Tutorial $\rightarrow$ shiny.rstudio.com/tutorial
		\item Articles $\rightarrow$ shiny.rstudio.com/articles
		\item NB `cf. *.html' in the rest of the slides are shiny.rstudio.com pages
	\end{itemize}

	 \vspace*{\baselineskip} % --------------------------------------------------


	Google + Stack Overflow 
}
