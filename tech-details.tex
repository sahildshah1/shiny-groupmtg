

\frame{
	\frametitle{Example}
	








}



\frame{
	\frametitle{Structure of a Shiny app}


	Structure of a Shiny app: two R scripts saved together in a directory
	\begin{itemize}
		\item Each app needs its own directory
		\item \texttt{ui.R} : controls layout and appearance of app
		\item \texttt{server.R}: contains instructions needed to build app
	\end{itemize}

	% http://shiny.rstudio.com/
	% http://shiny.rstudio.com/tutorial/lesson1/
	


}





\frame{
	\frametitle{Display reactive output}

	Create reactive output with two step process:
	\begin{enumerate}
		\item Add an R object to ui.R
		\item Tell Shiny how to build the object in server.R. The
		object will be reactive if the code that builds it calls a widget
		value. 
	\end{enumerate}

	Add an R object to UI: Output functions. 
	Place inside panelFunctions like you did with widgets
	\begin{itemize}
		\item Family of functions turn R objects into output
		\item eg. plotOutput, 
	\end{itemize}


	http://shiny.rstudio.com/tutorial/lesson4/
	
}


\frame{
	\frametitle{Display reactive output}

	Provide R code to build the object
	\begin{itemize}
		\item Providing R code that builds the object 
		\item Element name should match the name of the reactive element
		\item Each entry to output should contain the output of one of Shiny’s render* function
		\item Shiny will automatically make an object reactive if the object uses an input value
	\end{itemize}


	http://shiny.rstudio.com/tutorial/lesson4/

}


\frame{
	\frametitle{Execution}

	Where you put code in server.R determines how many times run
	\begin{itemize}
		\item Outside of \texttt{shinyServer{}} : Run once (library,load,source)

		\item In shinyServer run once a user visits app: eg. object records session info

		\item  render : Run each time user changes a widget 

	\end{itemize}


	http://shiny.rstudio.com/tutorial/lesson5/

}



