

\frame{
	\frametitle{'Hello Shiny' Example}

	\begin{block}{View example online }
        http://shiny.rstudio.com/tutorial/lesson1/
    \end{block}

	
	 \vspace*{\baselineskip} % --------------------------------------------------


	 \begin{center}OR\end{center}
	 \vspace*{\baselineskip} % --------------------------------------------------


	 \begin{block}{Run example in R}
	 	library(shiny) \\
        runExample("01\_hello") 
    \end{block}

	 

}


\frame{
	\frametitle{Structure of a Shiny app}

	Two R scripts saved together in a directory
	\begin{itemize}
		\item Each app needs its own directory
		\item \texttt{ui.R} : controls layout and appearance of app
		\item \texttt{server.R}: contains instructions needed to build app
	\end{itemize}

	% http://shiny.rstudio.com/
	% http://shiny.rstudio.com/tutorial/lesson1/

}


\frame{
	\frametitle{ui: Application layout}

	`Hello Shiny' example layout 
	\begin{itemize}
		\item \texttt{fluidPage}, \texttt{sidebarLayout} $\rightarrow$ Define layout
		\item Customize layout $\rightarrow$ cf. articles/layout-guide.html
	\end{itemize}


	% 	http://shiny.rstudio.com/articles/layout-guide.html

}

\frame{
	\frametitle{ui: widgets \& R objects}



	\textbf{Place widgets  and R objects  in layout}
	\vspace*{\baselineskip} % --------------------------------------------------

	\begin{itemize}
		\item widgets (eg slider) $\rightarrow$ cf. tutorial/lesson3
		\begin{itemize}
			\item \*Input functions create widgets 
			\item Access widget value in R object with the \texttt{label} arg
		\end{itemize}

		\item R object (eg plot) $\rightarrow$ cf. tutorial/lesson4
		\begin{itemize}
			\item \*Output functions create R objects
			\item The arg \*Output has to match label in \texttt{server.R}
		\end{itemize}

	\end{itemize}



}


\frame{
	\frametitle{server: R code that builds objects}

	\textbf{Build R objects in shinyServer function}
	\vspace*{\baselineskip} % --------------------------------------------------

	\begin{itemize}
		\item The output `list' stores R objects
			\begin{itemize}
				\item render\* function: instructions to create object $\rightarrow$ tutorial/lesson4
				\item Element name labels R object and should match argument of 
					  \*Output function in \texttt{ui.R}
			\end{itemize}
		\item The input `list' stores values of widgets
		\begin{itemize}
			\item Call a widget value inside a render function using the label
			  of the widget (Input function argument)
		\end{itemize}

	\end{itemize}


}

\frame{
	\frametitle{server: Execution of commands }

	Where you put code in server.R determines how many times run

	Execution of commands in server $\rightarrow$ cf. tutorial/lesson5/
	\vspace*{\baselineskip} % --------------------------------------------------


	\begin{itemize}
	
		\item \textbf{Outside of shinyServer}: Run once (eg. library, load, source)

		\item \textbf{In shinyServer}: Run once a user visits app (eg. record session info)

		\item \textbf{In render}: Run each time user changes a widget 
		
	\end{itemize}


}



